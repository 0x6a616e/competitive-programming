\documentclass[11pt]{article}
\usepackage[margin=2cm]{geometry}
\usepackage{graphicx}
\usepackage{listings}
\usepackage{mathtools}

\lstset{language=C++}
\setlength{\parindent}{0pt}

\begin{document}

\tableofcontents

\newpage

\section{C++}

\subsection{Plantilla básica}

\begin{lstlisting}
#include <bits/stdc++.h>

using namespace std;

int main() {
    ios::sync_with_stdio(0);
    cin.tie(0);

    return 0;
}
\end{lstlisting}

\subsection{Compilación}

\begin{lstlisting}[language=bash]
g++ -O2 -Wall -std=c++11 test.cpp
\end{lstlisting}

\subsection{Manejo de archivos}

\begin{lstlisting}
freopen("input.txt", "r", stdin);
freopen("output.txt", "w", stdout);
\end{lstlisting}

\subsection{Tipos de dato y sus rangos}

\begin{center}
    \begin{tabular}{ |l|l|l| }
         \hline
         Tipo de dato & Tamaño (bytes) & Rango \\
         \hline
         short int & 2 & -32,768 to 32,767 \\
         unsigned short int & 2 & 0 to 65,535 \\
         unsigned int & 4 & 0 to 4,294,967,295 \\
         int & 4 & -2,147,483,648 to 2,147,483,647 \\
         long int & 4 & -2,147,483,648 to 2,147,483,647 \\
         unsigned long int & 4 & 0 to 4,294,967,295 \\
         long long int & 8 & -(2\textasciicircum63) to (2\textasciicircum63)-1 \\
         unsigned long long int  & 8 & 0 to 18,446,744,073,709,551,615 \\
         signed char & 1 & -128 to 127 \\
         unsigned char & 1 & 0 to 255 \\
         float & 4 & -3.4×10\textasciicircum38 to 3.4×10\textasciicircum38 \\
         double & 8 & -1.7×10\textasciicircum308 to 1.7×10\textasciicircum308 \\
         long double & 12 & -1.1×10\textasciicircum4932 to 1.1×10\textasciicircum4932 \\
         \hline
    \end{tabular}
\end{center}

Algo a considear es que \textit{long long} tiene un prefijo.

\begin{lstlisting}
long long x = 123456789123456789LL;
\end{lstlisting}

Pasa algo parecido a la división entera, el resultado de multiplicar dos enteros es un entero. 

\subsection{Números flotantes}

Existen los errores de precisión, para tratar de contrarestarlos se puede usar esta lógica.
\begin{lstlisting}
if (abs(a - b) < 1e-9) {
    // a and b are equal
}
\end{lstlisting}

\subsection{Estimación de eficiencia}

\begin{center}
    \begin{tabular}{ | l | l | }
        \hline
        Tamaño de entrada & Peor complejidad \\
        \hline
        $n \leq 10$ & $O(n!)$ \\
        $n \leq 20$ & $O(2^n)$ \\
        $n \leq 500$ & $O(n^3)$ \\
        $n \leq 5000$ & $O(n^2)$ \\
        $n \leq 10^6$ & $O(n log(n))$ o $O(n)$ \\
        $n$ es grande & $O(1)$ o $O(log(n))$ \\
        \hline
    \end{tabular}
\end{center}

\section{Funciones y estructuras útiles}

\subsection{Manejo de fechas}

\begin{lstlisting}
struct Date {
    int day, month, year;
    vector<vector<int>> calendar = {
        { 0, 0 },
        { 31, 31 },
        { 28, 29 },
        { 31, 31 },
        { 30, 30 },
        { 31, 31 },
        { 30, 30 },
        { 31, 31 },
        { 31, 31 },
        { 30, 30 },
        { 31, 31 },
        { 30, 30 },
        { 31, 31 },
    };
    map<string, vector<int>> zodiacs = {
        { "aquarius" , { 1, 21, 2, 19 } },
        { "pisces" , { 2, 20, 3, 20 } },
        { "aries" , { 3, 21, 4, 20 } },
        { "taurus" , { 4, 21, 5, 21 } },
        { "gemini" , { 5, 22, 6, 21 } },
        { "cancer" , { 6, 22, 7, 22 } },
        { "leo" , { 7, 23, 8, 21 } },
        { "virgo" , { 8, 22, 9, 23 } },
        { "libra" , { 9, 24, 10, 23 } },
        { "scorpio" , { 10, 24, 11, 22 } },
        { "sagittarius" , { 11, 23, 12, 22 } },
        { "capricorn" , { 12, 23, 1, 20 } },
    };

    Date(int dd, int mm, int yy) {
        day = dd;
        month = mm;
        year = yy;

        add_days(0);
    }

    void add_days(int days_to_add) {
        day += days_to_add;
        int leap = is_leap();

        while (day > calendar[month][leap]) {
            day -= calendar[month][leap];
            month++;
            if (month > 12) {
                month = 1;
                year++;
                leap = is_leap();
            }
        }
    }

    int is_leap() {
        if (year % 400 == 0) {
            return 1;
        }
        if (year % 4 == 0 && year % 100 != 0) {
            return 1;
        }
        return 0;
    }

    string zodiac() {
        int di, df, mi, mf;
        map<string, vector<int>>::iterator it;
        for (it = zodiacs.begin(); it != zodiacs.end(); it++) {
            di = it->second[1];
            mi = it->second[0];
            df = it->second[3];
            mf = it->second[2];

            if ((month == mi && day >= di) || (month == mf && day <= df))
                return it->first;
        }
        return "";
    }
};
\end{lstlisting}

\subsection{Imprimir tiempo}

Aquí hay una función para imprimir en formato HH:MM:SS a partir de la cantidad de segundos.

\begin{lstlisting}
void printTime(int seconds) {
    int hours = seconds / 3600;
    int minutes = (seconds % 3600) / 60;
    int secs = seconds % 60;
        
    cout << setfill('0') << setw(2) << hours << ":"
        << setfill('0') << setw(2) << minutes << ":"
        << setfill('0') << setw(2) << secs << '\n';
}
\end{lstlisting}

\subsection{Exponenciación binaria mod n}

\[ a^b \bmod m \]

\begin{lstlisting}
long long binpow(long long a, long long b, long long m) {
    a %= m;
    long long res = 1;
    while (b > 0) {
        if (b & 1)
        res = res * a % m;
        a = a * a % m;
        b >>= 1;
    }
    return res;
}
\end{lstlisting}

\section{Matemáticas}

\subsection{Sucesión simple}

\[ 1 + 2 + 3 + \dots + n = \frac{n(n + 1)}{2} \]

\[ 1^2 + 2^2 + 3^2 + \dots + n^2 = \frac{n(n + 1)(2n + 1)}{6} \]

\[ 1^3 + 2^3 + 3^3 + \dots + n^3 = \left[\frac{n(n + 1)}{2}\right]^2 \]

\subsection{Sucesión aritmetica}

Otra cosa es una sucesión aritmetica, donde la diferencia entre dos números cualesquiera es la misma.

\[ 3, 7, 11, 15 \]

En esta sucesión aumentan de 4 en 4. La formula se ve así.

\[ a + \dots + b = \frac{n(a + b)}{2} \]

\subsection{Sucesión geometrica}

Esta sucesión es una secuencia de números donde la proporción entre dos números consecutivos es la misma.

\[ a + ak + ak^2 + \dots + b = \frac{bk - a}{k - 1} \]

\subsection{Factorial}

Se puede definir iterativamente.

\[ \prod_{x=1}^{n} x = 1 \cdot 2 \cdot 3 \dots n \]

O recursivamente.

\[ 0! = 1 \]

\[ n! = n \cdot (n - 1)! \]

\subsection{Números de Fibonacci}

Se define recursivamente así.

\[ f(0) = 0 \]

\[ f(1) = 1 \]

\[ f(n) = f(n - 1) + f(n - 2) \]

Hay una forma de calcular el $n$-th número de Fibonacci en $O(n)$.

\begin{lstlisting}
int fib(int n) {
    int a = 0;
    int b = 1;
    for (int i = 0; i < n; i++) {
        int tmp = a + b;
        a = b;
        b = tmp;
    }
    return a;
}
\end{lstlisting}

Y hay una de hacerlo en $O(log(n))$.

\begin{lstlisting}
struct matrix {
    long long mat[2][2];
    matrix friend operator *(const matrix &a, const matrix &b) {
        matrix c;
        for (int i = 0; i < 2; i++) {
            for (int j = 0; j < 2; j++) {
                c.mat[i][j] = 0;
                for (int k = 0; k < 2; k++) {
                    c.mat[i][j] += a.mat[i][k] * b.mat[k][j];
                }
            }
        }
        return c;
    }
};

matrix matpow(matrix base, long long n) {
    matrix ans{
        {
            { 1, 0 },
            { 0, 1 }
        }
    };
    while (n) {
        if(n & 1)
            ans = ans * base;
        base = base * base;
        n >>= 1;
    }
    return ans;
}

long long fib(int n) {
    matrix base{
        {
            { 1, 1 },
            { 1, 0 }
        }
    };
    return matpow(base, n).mat[0][1];
}
\end{lstlisting}

\subsection{Logaritmos}

Propiedades de los logaritmos.

\[ log_k(x) = a \implies k^a = x \]

\[ log_k(ab) = log_k(a) + log_k(b) \]

\[ log_k(x^n) = n \cdot log_k(x) \]

\[ log_k(\frac{a}{b}) = log_k(a) - log_k(b) \]

Otra propiedad interesante es que la cantidad de digitos de un entero \textit{x} en base \textit{b} es:

\[ log_b(x) + 1 \]

\subsection{Promedio}

Se puede sumar o restar un número a un promedio ya calculado sin tener que recalcular la suma original.

\[ s = \frac{a_1 + \dots + a_n}{n} \]

\[ s' = \frac{a_1 + \dots + a_n + a_{n + 1}}{n + 1} = \frac{ns + a_{n + 1}}{n + 1} = \frac{(n + 1)s + a_{n + 1}}{n + 1} - \frac{s}{n + 1} = s + \frac{a_{n + 1} - s}{n + 1} \]

\[ s'' = \frac{a_1 + \dots + a_{n - 1}}{n - 1} = \frac{ns - a_{n}}{n - 1} = \frac{(n - 1)s + a_{n}}{n - 1} + \frac{s}{n - 1} = s + \frac{s - a_{n}}{n - 1} \]

\subsection{Generar números primos}

La criba de Eratóstenes es una forma de calcular los números primos en el intervalo de $[1 ; n]$ con complejidad $O(n \cdot log(log(n)))$.

Su implementación es:

\begin{lstlisting}
int n;
vector<bool> is_prime(n + 1, true);
is_prime[0] = is_prime[1] = false;
for (int i = 2; i * i <= n; i++) {
    if (is_prime[i]) {
        for (int j = i * i; j <= n; j += i)
            is_prime[j] = false;
    }
}
\end{lstlisting}

\subsection{Prueba de primalidad}

Forma deterministica de comprobar si un número de hasta 64 bits es primo. 

\begin{lstlisting}
using u64 = uint64_t;
using u128 = __uint128_t;

u64 binpower(u64 base, u64 e, u64 mod) {
    u64 result = 1;
    base %= mod;
    while (e) {
        if (e & 1)
            result = (u128)result * base % mod;
        base = (u128)base * base % mod;
        e >>= 1;
    }
    return result;
}

bool check_composite(u64 n, u64 a, u64 d, int s) {
    u64 x = binpower(a, d, n);
    if (x == 1 || x == n - 1)
        return false;
    for (int r = 1; r < s; r++) {
        x = (u128)x * x % n;
        if (x == n - 1)
            return false;
    }
    return true;
};

bool MillerRabin(u64 n) { // Main func
    if (n < 2)
        return false;

    int r = 0;
    u64 d = n - 1;
    while ((d & 1) == 0) {
        d >>= 1;
        r++;
    }

    for (int a : {2, 3, 5, 7, 11, 13, 17, 19, 23, 29, 31, 37}) {
        if (n == a)
            return true;
        if (check_composite(n, a, d, r))
            return false;
    }
    return true;
}
\end{lstlisting}

\subsection{Factores primos}

La forma más simple de obtener los factores primos de un número es pregenerando los primos desde $1$ hasta $\sqrt{n}$ y probando con cada uno de ellos.

\begin{lstlisting}
vector<long long> primes;

vector<long long> trial_division4(long long n) {
    vector<long long> factorization;
    for (long long d : primes) {
        if (d * d > n)
            break;
        while (n % d == 0) {
            factorization.push_back(d);
            n /= d;
        }
    }
    if (n > 1)
        factorization.push_back(n);
    return factorization;
}
\end{lstlisting}

\subsection{Máximo común divisor}

\[ \gcd(a, b) = \begin{cases}a,&\text{if }b = 0 \\ \gcd(b, a \bmod b),&\text{otherwise.}\end{cases} \]

Una implementación recursiva:

\begin{lstlisting}
int gcd(int a, int b) {
    return b ? gcd(b, a % b) : a;
}
\end{lstlisting}

Y una iterativa:

\begin{lstlisting}
int gcd(int a, int b) {
    while (b) {
        a %= b;
        swap(a, b);
    }
    return a;
}
\end{lstlisting}

\subsection{Mínimo común múltiplo}

\[ lcm(a, b) = \frac{a \cdot b}{\gcd(a, b)} \]

Y aquí su implementación:

\begin{lstlisting}
int lcm(int a, int b) {
    return a / gcd(a, b) * b;
}
\end{lstlisting}

\subsection{Ceil/Floor}

Estas funciones a veces pueden tener problemas de precisión, por lo que puede ser mejor aplicarlas de forma que siempre se mantienen enteros.

Floor suele ser innecesario porque ese es el comportamiento por defecto de C++ al dividir enteros.

Ceil se puede sustituir por:

\[ \lceil \frac{a}{b} \rceil = \lfloor \frac{a + b - 1}{b} \rfloor \]

Esta igualdad se cumple para enteros positivos, para enteros negativos no estoy muy seguro.

\subsection{Cantidad de múltiplos de un número}

Para problemas donde se quiera saber por ejemplo, cuantos múltiplos de 2 hay del 1 al 100 se puede responder dividiendo este número entre la base.

Para el caso de múltiplos de dos números, por ejemplo, cuantos números hay del 1 al 100 que sean múltiplos de 2 y de 5 se usa el mínimo común múltiplo como base.

\section{Ordenamiento}

Existen las funciones \textit{sort} y \textit{stable\_sort}.

\begin{lstlisting}
vector<int> v = { 4, 2, 5, 3, 5, 8, 3 };
sort(v.begin(), v.end());
\end{lstlisting}

Después de este ordenamiento, el contenido del vector será:

\[ [ 2, 3, 3, 4, 5, 5, 8 ] \]

Por defecto se ordena de menor a mayor, pero una forma de implementar un orden inverso es:

\begin{lstlisting}
sort(v.rbegin(), v.rend());
\end{lstlisting}

\subsection{Estructuras definidas}

Las estructuras que nosotros hagamos no tienen un operador de comparación automaticamente. Este operador se puede definir dentro de la estructura como una función de nombre \textit{operator<} cuyo parametro debe ser un elemento del mismo tipo, debe devolver \textit{true} si el elemento es más pequeño que el parametro y \textit{falso} si no es así. Un ejemplo:

\begin{lstlisting}
struct P {
    int x, y;
    bool operator<(const P &p) {
        if (x != p.x)
            return x < p.x;
        return y < p.y;
    }
};
\end{lstlisting}

\subsection{Con funciones personalizadas}

También se puede dar como parametro una función externa para que se use dentro del \textit{sort}. Por ejemplo:

\begin{lstlisting}
bool comp(string &a, string &b) {
    if (a.size() != b.size())
        return a.size() < b.size();
    return a < b;
}
\end{lstlisting}

Ahora usando esa función para un vector de strings:

\begin{lstlisting}
sort(v.begin(), v.end(), comp);
\end{lstlisting}

\subsection{Permutaciones}

Una forma simple de pasar por todas las permutaciones de algo es con \textit{next\_permutation}, pero primero se ordena lo que se vaya a permutar. Ejemplo:

\begin{lstlisting}
letras = "cba";
sort(letras.begin(), letras.end());
do {
    cout << letras << endl;
} while (next_permutation(letras.begin(), letras.end()));
\end{lstlisting}

\subsection{Ordenamiento parcial}

En caso de que no se necesite ordenar todos los datos se pueden hacer dos tipos de ordenamientos parciales.

Para ordenar de modo que los primeros N elementos esten ordenados se puede usar:

\begin{lstlisting}
partial_sort( RandomIt first, RandomIt middle, RandomIt last );
\end{lstlisting}

Esta función se asegura que los elementos hasta \textit{middle} esten ordenados, OJO \textit{middle} no es un valor, es un iterador, así que no ordena los menores a \textit{middle} sino los \textit{middle - first}.

Otra forma es si se quiere que el enesimo elemento del arreglo este ordenado, para eso se usa:

\begin{lstlisting}
nth_element( RandomIt first, RandomIt nth, RandomIt last );
\end{lstlisting}

Que igualmente recibe tres iteradores.

\subsection{Búsqueda binaria manual}

Hay dos formas de implementar la búsqueda binaria por nuestra cuenta.

\begin{lstlisting}
int a = 0, b = n - 1;
while (a <= b) {
    int k = (a + b) / 2;
    if (array[k] == x) {
        // x found at index k
    }
    if (array[k] > x)
        b = k - 1;
    else
        a = k + 1;
}
\end{lstlisting}

Y.

\begin{lstlisting}
int k = 0;
for (int b = n / 2; b >= 1; b /= 2) {
    while (k + b < n && array[k + b] <= x)
        k += b;
}
if (array[k] == x) {
    // x found at index k
}
\end{lstlisting}

\subsection{Búsqueda binaria con C++}

La libreria estandar de C++ contiene las siguientes funciones que estan basadas en búsqueda binaria y por lo tanto funcionan en tiempo logaritmico:

\begin{description}
    \item[lower\_bound] Devuelve un apuntador al primer elemento que vale al menos x.
    \item[upper\_bound] Devuelve un apuntador al primer elemento que vale más que x.
    \item[equal\_range] Devuelve los dos apuntadores de arriba.
\end{description}

Estas funciones asumen que el arreglo esta ordenado. Si no encuentran el
elemento devuelven un apuntador pasado el último elemento.

\section{Manipulación de bits}

\subsection{Operadores binarios}

\begin{description}
    \item[$ \& $] AND
    \item[$ | $] OR
    \item[\^{}] XOR
    \item[$ \sim $] NOT
\end{description}

Ejemplos:

\begin{verbatim}
n           = 01011000
n - 1       = 01010111
----------------------
n & (n - 1) = 01010000
\end{verbatim}

\begin{verbatim}
n           = 01011000
n - 1       = 01010111
----------------------
n | (n - 1) = 01011111
\end{verbatim}

\begin{verbatim}
n           = 01011000
n - 1       = 01010111
----------------------
n ^ (n - 1) = 00001111
\end{verbatim}

\begin{verbatim}
n         = 01011000
--------------------
~n        = 10100111
\end{verbatim}

\subsection{Desplazamientos}

\begin{description}
    \item[$ >> $] Desplaza el número a la derecha removiendo bits, es lo mismo que dividir a la mitad.
    \item[$ << $] Desplaza el número a la izquierda añadiendo bits vacios, es lo mismo a duplicar un número.
\end{description}

\subsection{Trucos útiles}

Se puede asignar, invertir o limpiar un bit usando las siguientes propiedades:

\begin{description}
    \item[$ n | (1 << x) $] Activa el $x$-th bit en el número n.
    \item[$ n $ \^{} $ (1 << x) $] Invierte el $x$-th bit en el número n.
    \item[$ n \& \sim (1 << x) $] Limpia el $x$-th bit del número n.
    \item[$ n \& (n - 1) $] Limpia el bit más a la derecha de n, se puede usar para saber si es una potencia.
\end{description}

\begin{verbatim}  

Se puede revisar si un bit $x$ esta activo con:

\begin{lstlisting}
bool is_set(unsigned int number, int x) {
    return (number >> x) & 1;
}
\end{lstlisting}

Se puede limpiar el bit más a la derecha de un número haciendo un AND con su
predecesor.
 
\end{lstlisting}
n           = 00110100
n - 1       = 00110011
----------------------
n & (n - 1) = 00110000
\end{lstlisting}

\begin{verbatim}
# Submascaras de una mascara

Dada una mascara $m$ se quieren iterar por todas sus submascaras, es decir,
mascaras $s$ en las que solo bits que se incluían en $m$ estan activos.

\begin{lstlisting}
for (int s=m; s; s=(s-1)&m)
    ... you can use s ...
\end{lstlisting}

Esto no incluye la submascara equivalente a 0.

# Manejo de números grandes

Para el manejo de números grandes se va a usar un arreglo donde se guarden sus
"digitos".

\begin{lstlisting}
// Base a usar
const int base = 1000 textit{ 1000 } 1000;

// Convertir de una string al vector de "digitos"
for (int i = (int) s.length(); i > 0; i -= 9) {
    if (i < 9) {
        a.push_back(atoi(s.substr(0, i).c_str()));
    } else {
        a.push_back(atoi(s.substr(i - 9, 9).c_str()));
    }
}

// Borrar leading 0, combiene hacerlo después de la mayoría de operaciones
while (a.size() > 1 && a.back() == 0) {
    a.pop_back();
}

// Suma a + b resultado en a
int carry = 0;
for (size_t i = 0; i < max(a.size(), b.size()) || carry; ++i) {
    if (i == a.size()) {
        a.push_back(0);
    }
    a[i] += carry + (i < b.size() ? b[i] : 0);
    carry = a[i] >= base;
    if (carry) {
        a[i] -= base;
    }
}

// Resta a - b y guarda en a
int carry = 0;
for (size_t i = 0; i < b.size() || carry; ++i) {
    a[i] -= carry + (i < b.size() ? b[i] : 0);
    carry = a[i] < 0;
    if (carry) {
        a[i] += base;
    }
}

// Multiplicar a por un entero b pequeño (b < base) y guardar en a
int carry = 0;
for (size_t i = 0; i < a.size() || carry; ++i) {
    if (i == a.size()) {
        a.push_back(0);
    }
    long long cur = carry + a[i] textit{ 1ll } b;
    a[i] = int (cur % base);
    carry = int (cur / base);
}

// Multiplicar a por un entero largo b y guardar en c
vector<int> c(a.size() + b.size());
for (size_t i = 0; i < a.size(); ++i) {
    for (int j = 0, carry = 0; j < (int) b.size() || carry; ++j) {
        long long cur = c[i + j] + a[i] textit{ 1ll } (j < (int) b.size() ? b[j] : 0) + carry;
        c[i + j] = int (cur % base);
        carry = int (cur / base);
    }
}

// División de a entre un entero b pequeño (b < base)
int carry = 0;
for (int i = (int) a.size() - 1; i >= 0; --i) {
    long long cur = a[i] + carry textit{ 1ll } base;
    a[i] = int (cur / b);
    carry = int (cur % b);
}

// Para imprimir
if (a.empty()) {
    cout << 0;
} else {
    cout << a.back();
}
        
for (int i = (int) a.size() - 2; i >= 0; --i) {
    cout << setw(9) << setfill('0') << a[i];
}
\end{lstlisting}

# Estructuras de datos

\subsection{Policy Based Data Structures}

Es una estructura muy útil, es básicamente un textit{set} (con inserción y borrado en
$O(log(n))$) pero con índices.

\begin{lstlisting}
#include <ext/pb_ds/assoc_container.hpp>

using namespace __gnu_pbds;

typedef tree<
    pair<unsigned long long, int>,
    null_type,
    less<pair<unsigned long long, int>>,
    rb_tree_tag,
    tree_order_statistics_node_update
> indexed_set;
\end{lstlisting}

Esta implementación es para un multiset, por eso se almacena un textit{pair}, pero si
se necesita que no haya repeticiones se sustituye el textit{pair} por el tipo de dato.
También esta ordenado de menor a mayor, si se necesita lo opuesto se cambia el
textit{less}.

\begin{lstlisting}
indexed_set is;

is.order_of_key(key); // Regresa el índice que esa key tendría dentro del set, exista o no
is.find_by_order(order); // Regresa un apuntador al índice que se uso como parametro
\end{lstlisting}

\subsection{Minimum stack / Minimum queue}

Aquí hay modificaciones al stack y a la fila para además de tener sus
características poder acceder al menor elemento en $O(1)$.

#\subsection{Minimum stack}

\begin{lstlisting}
stack<pair<int, int>> st;

// Agregar elemento
int new_min = st.empty() ? new_elem : min(new_elem, st.top().second);
st.push({ new_elem, new_min });

// Remover elemento
int removed_element = st.top().first;
st.pop();

// Encontrar el mínimo
int minimum = st.top().second;
\end{lstlisting}

#\subsection{Minimum queue}

\begin{lstlisting}
stack<pair<int, int>> s1, s2;

// Encontrar el mínimo
if (s1.empty() || s2.empty()) 
    minimum = s1.empty() ? s2.top().second : s1.top().second;
else
    minimum = min(s1.top().second, s2.top().second);

// Añadir elemento
int minimum = s1.empty() ? new_element : min(new_element, s1.top().second);
s1.push({new_element, minimum});

// Remover elemento
if (s2.empty()) {
    while (!s1.empty()) {
        int element = s1.top().first;
        s1.pop();
        int minimum = s2.empty() ? element : min(element, s2.top().second);
        s2.push({element, minimum});
    }
}
int remove_element = s2.top().first;
s2.pop();
\end{lstlisting}

\subsection{Grafos}

Hay tres formas principales de representar un grafo.

Matriz de adyacencias
: Es buena elección si constantemente se necesita revisar si dos vertices estan
conectados en un grafo denso. Pero no se recomienda para grafos grandes y
dispersos porque requiere $O(V^2)$ espacio y habría mucho desperdiciado con
celdas en blanco. Usualmente el límite de vertices para una matriz de
adyacencias en una competencia sería de 1000, más de esos y ya se vuelve una
mala idea. También se necesita $O(V)$ para enumerar todos los vecinos de un
vertice.

Lista de adyacencias
: Es una forma más eficiente de representar un grafo, se recomienda sea la
primera opción a considerar al encontrarse con problemas de grafos. Su espacio
es $O(V + E)$ y se suele representar como un vector de vectores o como un vector
de vectores de pares.

Lista de arcos
: Es otra forma en donde se usa un vector de trios (para grafos con peso) o uno
de pares si no hay peso y cada entrada representa un arco, tiene espacio
$O(E)$ y aunque dificulta el ver los vecinos de cierto nodo puede simplificar
ciertos algoritmos.

\subsection{Union-Find Disjoint Set}

Es una estructura de datos que permite de forma eficiente determinar que nodos
pertenencen al mismo set, así como poder combinar sets. Su implementacion:

\begin{lstlisting}
struct UnionFind {
    vector<int> parent, rank;

    unionFind(int N) {
        rank.assign(N, 0);
        parent.assign(N, 0);
        for (int i = 0; i < N; ++i) parent[i] = i;
    }

    int findSet(int i) {
        if (parent[i] == i) return i;

        parent[i] = findSet(parent[i]);

        return parent[i];
    }

    bool isSameSet(int i, int j) { return findSet(i) == findSet(j); }

    void unionSet(int i, int j) {
        if (!isSameSet(i, j)) {
            int x = findSet(i);
            int y = findSet(j);

            if (rank[x] > rank[y])
                parent[y] = x;
            else {
                parent[x] = y;
                if (rank[x] == rank[y]) ++rank[y];
            }
        }
    }
};
\end{lstlisting}

\subsection{Segment Tree}

El árbol de segmentos es una estructura de datos que sirve para responder de
forma eficiente consultas de rangos en arreglos de números que pueden cambiar.

Primero una implementación para obtener el mínimo en cierto rango:

\begin{lstlisting}
struct SegmentTree {
    vector<int> st, A;
    int n;

    int left(int p) { return p << 1; }

    int right(int p) { return (p << 1) + 1; }

    void build(int p, int L, int R) {
        if (L == R)
            st[p] = L;
        else {
            build(left(p), L, (L + R) / 2);
            build(right(p), (L + R) / 2 + 1, R);
            int p1 = st[left(p)];
            int p2 = st[right(p)];
            st[p] = (A[p1] <= A[p2]) ? p1 : p2;
        }
    }

    int rmq(int p, int L, int R, int i, int j) {
        if (i > R || j < L) return -1;

        if (L >= i && R <= j) return st[p];

        int p1 = rmq(left(p), L, (L + R) / 2, i, j);
        int p2 = rmq(right(p), (L + R) / 2 + 1, R, i, j);

        if (p1 == -1) return p2;

        if (p2 == -1) return p1;

        return (A[p1] <= A[p2]) ? p1 : p2;
    }

    void pointUpdate(int p, int L, int R, int i, int v) {
        if (L == R) return;

        if (i >= L && i <= R) {
            if (v <= A[st[p]]) st[p] = i;
            
            pointUpdate(left(p), L, (L + R) / 2, i, v);
            pointUpdate(right(p), (L + R) / 2 + 1, R, i, v);
        }
    }

    SegmentTree(vector<int> &_A) {
        A = _A;
        n = (int)A.size();
        st.assign(4 * n, 0);
        build(1, 0, n - 1);
    }

    int rmq(int i, int j) { return rmq(1, 0, n - 1, i, j); }

    void pointUpdate(int i, int v) {
        A[i] = v;
        pointUpdate(1, 0, n - 1, i, v);
    }

    void rangeUpdate(int i, int j, int v) {
        for (int k = j; k >= i; --k) pointUpdate(k, v);
    }
};
\end{lstlisting}

Ahora una implementación para la suma de los elementos en cierto rango:

\begin{lstlisting}
struct SegmentTree {
    vector<int> st, A;
    vector<bool> marked;
    int n;

    int left(int p) { return p << 1; }

    int right(int p) { return (p << 1) + 1; }

    void build(int p, int L, int R) {
        if (L == R)
            st[p] = A[L];
        else {
            build(left(p), L, (L + R) / 2);
            build(right(p), (L + R) / 2 + 1, R);
            int p1 = st[left(p)];
            int p2 = st[right(p)];
            st[p] = p1 + p2;
        }
    }

    int rsq(int p, int L, int R, int i, int j) {
        if (i > R || j < L) return 0;

        if (L >= i && R <= j) return st[p];

        push(p, L, R);

        int p1 = rsq(left(p), L, (L + R) / 2, i, j);
        int p2 = rsq(right(p), (L + R) / 2 + 1, R, i, j);

        return p1 + p2;
    }

    void push(int p, int L, int R) {
        if (L == R) return;
        if (marked[p]) {
            int tm = (L + R) / 2;
            int v = st[p] / (R - L + 1);
            st[left(p)] = v * (tm - L + 1);
            st[right(p)] = v * (R - tm);
            marked[left(p)] = marked[right(p)] = true;
            marked[p] = false;
        }
    }

    void rangeUpdate(int p, int L, int R, int i, int j, int v) {
        if (i > R || j < L) return;
        if (L >= i && R <= j) {
            st[p] = v * (R - L + 1);
            marked[p] = true;
            return;
        }
        push(p, L, R);
        rangeUpdate(left(p), L, (L + R) / 2, i, j, v);
        rangeUpdate(right(p), (L + R) / 2 + 1, R, i, j, v);
        st[p] = st[left(p)] + st[right(p)];
    }

    SegmentTree(vector<int>& _A) {
        A = _A;
        n = (int)A.size();
        st.assign(4 * n, 0);
        marked.assign(4 * n, false);
        build(1, 0, n - 1);
    }

    int rsq(int i, int j) { return rsq(1, 0, n - 1, i, j); }

    void rangeUpdate(int i, int j, int v) { rangeUpdate(1, 0, n - 1, i, j, v); }
};
\end{lstlisting}

\subsection{Fenwick Tree}

Muy similar al Segment Tree, por ahora creo que le daré prioridad al Segment
Tree en lugar de este pero quisiera explorarlo mejor después.

\begin{lstlisting}
struct FenwickTree {
    vector<int> ft;

    int LSOne(int S) { return (S & (-S)); }

    FenwickTree(int n) { ft.assign(n + 1, 0); }

    int rsq(int b) {
        int sum = 0;
        for (; b; b -= LSOne(b)) sum += ft[b];

        return sum;
    }

    int rsq(int a, int b) { return rsq(b) - (a == 1 ? 0 : rsq(a - 1)); }

    void adjust(int k, int v) {
        for (; k < (int)ft.size(); k += LSOne(k)) ft[k] += v;
    }
};
\end{lstlisting}

# Problemas de ejemplo

\subsection{Suma máxima de subarreglo}

Aquí se habla de un problema clásico cuya solución más textit{directa} es O(n^3),
pero que pensandola mejor se puede lograr reducir a O(n). Dado un arreglo de n
números, hay que calcular cual es la suma máxima de una subsección, osea la
suma más grande de una secuencia consecutiva de valores en el arreglo, esto se
vuelve más interesante si el arreglo puede tener números negativos. Aquí hay un
ejemplo.

| -1 | 2 | 4 | -3 | 5 | 2 | -5 | 2 |

El siguiente subarreglo produce una suma de 10:

| -1 | 2 | 4 | -3 | 5 | 2 | -5 | 2 |
|----|---|---|----|---|---|----|---|
|    | ^ | ^ |  ^ | ^ | ^ |    |   |

Asumimos que un subarreglo vacio esta permitido, por lo que la suma máxima
siempre será al menos 0. Ahora, para resolverlo en O(n) partimos de la idea de
que para cada posición queremos calcular la suma máxima posible hasta ahí,
entonces la del arreglo completo será la mayor de todas esas sumas. Consideramos
dos posibilidades para la máxima suma del arreglo que termina en la posición textit{k}:

1. El subarreglo solo contiene el elemento de la posición k.
2. El subarreglo consiste de un subarreglo que termina en la posición k - 1,
seguido del elemento en la posición k.

En el segundo caso, dado que queremos encontrar un subarreglo con la máxima
suma, el subarreglo que termina en la posición k - 1 debería tener también la
máxima suma para ese punto. Entonces podemos resolver el problema al calcular
la máxima suma de subarreglos para cada posición de izquierda a derecha. El
siguiente código demuestra una implementación de este algoritmo:

\begin{lstlisting}
int best = 0, sum = 0;
for (int k = 0; k < n; k++) {
    sum = max(array[k],sum+array[k]);
    best = max(best,sum);
}
cout << best << '\n';
\end{lstlisting}

Este algoritmo solo tiene un ciclo, por lo que la complejidad es O(n). Esta es
la mejor complejidad posible, ya que cualquier algoritmo para este problema
tiene que analizar todos los datos al menos una vez.

\subsection{Segment Tree para arreglo binario}

Hay un arreglo binario donde se necesita saber cuantos 1's hay en determinado
rango, también es necesario poder actualizar un rango así como poder invertir
un rango (cambiar 0 -> 1 y 1 -> 0).

\begin{lstlisting}
struct SegmentTree {
    vector<int> st, A, marked;
    int n;

    int left(int p) { return p << 1; }

    int right(int p) { return (p << 1) + 1; }

    void build(int p, int L, int R) {
        if (L == R)
            st[p] = A[L];
        else {
            build(left(p), L, (L + R) / 2);
            build(right(p), (L + R) / 2 + 1, R);
            int p1 = st[left(p)];
            int p2 = st[right(p)];
            st[p] = p1 + p2;
        }
    }

    int rsq(int p, int L, int R, int i, int j) {
        if (i > R || j < L) return 0;

        if (L >= i && R <= j) return st[p];

        push(p, L, R);

        int p1 = rsq(left(p), L, (L + R) / 2, i, j);
        int p2 = rsq(right(p), (L + R) / 2 + 1, R, i, j);

        return p1 + p2;
    }

    void push(int p, int L, int R) {
        if (L == R) return;
        int tm = (L + R) / 2;
        if (marked[p] == 1) {
            int v = (st[p] ? 1 : 0);
            st[left(p)] = v * (tm - L + 1);
            st[right(p)] = v * (R - tm);
            marked[left(p)] = marked[right(p)] = 1;
            marked[p] = 0;
        } else if (marked[p] == 2) {
            push(left(p), L, tm);
            push(right(p), tm + 1, R);
            st[left(p)] = (tm - L + 1) - st[left(p)];
            st[right(p)] = (R - tm) - st[right(p)];
            marked[left(p)] = marked[right(p)] = 2;
            marked[p] = 0;
        }
    }

    void invertRange(int p, int L, int R, int i, int j) {
        if (i > R || j < L) return;
        push(p, L, R);
        if (L >= i && R <= j) {
            st[p] = (R - L + 1) - st[p];
            marked[p] = 2;
            return;
        }
        invertRange(left(p), L, (L + R) / 2, i, j);
        invertRange(right(p), (L + R) / 2 + 1, R, i, j);
        st[p] = st[left(p)] + st[right(p)];
    }

    void rangeUpdate(int p, int L, int R, int i, int j, int v) {
        if (i > R || j < L) return;
        if (L >= i && R <= j) {
            st[p] = v * (R - L + 1);
            marked[p] = 1;
            return;
        }
        push(p, L, R);
        rangeUpdate(left(p), L, (L + R) / 2, i, j, v);
        rangeUpdate(right(p), (L + R) / 2 + 1, R, i, j, v);
        st[p] = st[left(p)] + st[right(p)];
    }

    SegmentTree(vector<int>& _A) {
        A = _A;
        n = (int)A.size();
        st.assign(4 * n, 0);
        marked.assign(4 * n, 0);
        build(1, 0, n - 1);
    }

    int rsq(int i, int j) { return rsq(1, 0, n - 1, i, j); }

    void rangeUpdate(int i, int j, int v) { rangeUpdate(1, 0, n - 1, i, j, v); }

    void invertRange(int i, int j) { invertRange(1, 0, n - 1, i, j); }
};
\end{lstlisting}
\end{verbatim}
\end{document}
